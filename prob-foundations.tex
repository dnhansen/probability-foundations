% Document setup
\documentclass[article, a4paper, 11pt, oneside]{memoir}
\usepackage[utf8]{inputenc}
\usepackage[T1]{fontenc}
\usepackage[UKenglish]{babel}

% Document info
\newcommand\doctitle{Foundations of probability theory}
\newcommand\docauthor{Danny Nygård Hansen}

% Formatting and layout
\usepackage[autostyle]{csquotes}
\renewcommand{\mktextelp}{(\textellipsis\unkern)}
\usepackage[final]{microtype}
\usepackage{xcolor}
\frenchspacing
\usepackage{latex-sty/articlepagestyle}
\usepackage{latex-sty/articlesectionstyle}

% Fonts
\usepackage[largesmallcaps,partialup]{kpfonts}
\DeclareSymbolFontAlphabet{\mathrm}{operators} % https://tex.stackexchange.com/questions/40874/kpfonts-siunitx-and-math-alphabets
\linespread{1.06}
\let\mathfrak\undefined
\usepackage{eufrak}
\usepackage{inconsolata}
\usepackage{amssymb}

% Hyperlinks
\usepackage{hyperref}
\definecolor{linkcolor}{HTML}{4f4fa3}
\hypersetup{%
	pdftitle=\doctitle,
	pdfauthor=\docauthor,
	colorlinks,
	linkcolor=linkcolor,
	citecolor=linkcolor,
	urlcolor=linkcolor,
	bookmarksnumbered=true
}

% Equation numbering
\numberwithin{equation}{chapter}

% Footnotes
\footmarkstyle{\textsuperscript{#1}\hspace{0.25em}}

% Mathematics
\usepackage{latex-sty/basicmathcommands}
\usepackage{latex-sty/framedtheorems}
\usepackage{latex-sty/probabilitycommands}
\usepackage{tikz-cd}
\usetikzlibrary{babel}

% Lists
\usepackage{enumitem}
\setenumerate[0]{label=\normalfont(\arabic*)}

% Bibliography
\usepackage[backend=biber, style=authoryear, maxcitenames=2, useprefix]{biblatex}
\addbibresource{references.bib}

% Title
\title{\doctitle}
\author{\docauthor}


% Section style -- add to section style .sty?
\setsubsecheadstyle{\normalfont\itshape}


% Preimage -- to be added to mathcommands .sty
\newcommand{\preim}{^{-1}}


\newcommand{\calN}{\mathcal{N}}
\DeclarePairedDelimiter{\nhoodfilteraux}{(}{)}
% \newcommand{\nhoodfilter}[1]{\calN\nhoodfilteraux{#1}}
\newcommand{\nhoodfilter}[1]{\calN_{#1}}


\newcommand{\calU}{\mathcal{U}}
\newcommand{\calV}{\mathcal{V}}
\newcommand{\calW}{\mathcal{W}}
\newcommand{\calT}{\mathcal{T}}
\newcommand{\calB}{\mathcal{B}}
\newcommand{\calE}{\mathcal{E}}
\newcommand{\calF}{\mathcal{F}}
\newcommand{\calA}{\mathcal{A}}
\newcommand{\calD}{\mathcal{D}}
\newcommand{\calS}{\mathcal{S}}

\newcommand{\borel}[1]{\calB(#1)}
\DeclareMathOperator{\supp}{supp}

\let\oldP\P
\renewcommand{\P}{\mathbb{P}}

% Add to basicmathcommands.sty
\DeclarePairedDelimiter{\card}{\lvert}{\rvert}


\begin{document}

\maketitle

\chapter{Introduction}

In the usual measure-theoretical formulation of probability theory, the following result is a corollary of the Law of Large Numbers:

\begin{theorem}[The frequency interpretation of probability]
    Let $\rvar{X}, \rvar{X}_1, \rvar{X}_2, \ldots$ be i.i.d. real-valued random variables on a probability space $(\Omega, \calF, \P)$. For every $B \in \borel{\reals}$ we have
    %
    \begin{equation*}
        \P(\rvar{X} \in B)
            = \lim_{n\to\infty} \frac{
                \card{ \set{j \in \{1, \ldots, n\} }{ \rvar{X}_j(\omega) \in B } }
            }{
                n
            }
    \end{equation*}
    %
    for $\P$-almost all $\omega \in \Omega$.
\end{theorem}

\begin{proof}
    Thorbjørnsen Korollar~13.6.2. [Maybe reproduce the proof given LLN for completeness?]
\end{proof}
%
That is, given a sequence $(\rvar{X}_n)_{n\in\naturals}$, and one extra $\rvar{X}$, of i.i.d. random variables, the probability that $\rvar{X}$ lies in some Borel set $B$ can be thought of as the proportion of the $\rvar{X}_n$ that lie in $B$, as $n$ tends to infinity. In other words, probability is a measure of the \emph{frequency} with which an outcome of a random experiment obtains, if we repeat the experiment many times.

Whether or not this is the correct interpretation of probability as it occurs in the natural world we will not discuss here. Nonetheless the above result is an uncontroversial consequence of the theory, and it certainly aligns with our intuitive understanding of probability.

In this note we turn this result on its head and attempt to use it to motivate the formalisation of probability theory in terms of measure spaces. As we shall see, this is not entirely successful and will require some leaps that are not entirely justified by our conceptual grasp of probability.


\chapter{Metric Boolean algebras}

\begin{definition}[Boolean algebras]
    A \emph{Boolean algebra} is a structure $\langle B; \join, \meet, ', 0, 1 \rangle$ such that
    %
    \begin{enumdef}
        \item $\langle B; \join, \meet \rangle$ is a distributive lattice,
        \item $a \join 0 = a$ and $a \meet 1$ for all $a \in B$, and
        \item $a \join a' = 1$ and $a \meet a' = 0$ for all $a \in B$.
    \end{enumdef}
    %
    A \emph{measure} on a Boolean algebra $B$ is a map $\mu \colon B \to [0,\infty)$ such that $x \meet y = 0$ implies
    %
    \begin{equation}
        \label{eq:finite-additivity}
        \mu(x \join y)
            = \mu(x) + \mu(y)
    \end{equation}
    %
    for all $x,y \in B$.
\end{definition}
%
The property \cref{eq:finite-additivity} is called \emph{(finite) additivity} of $\mu$. Recall that the lattice structure induces an order $\leq$ on $B$ such that $x \leq y$ if and only if $x \join y = y$ for $x,y \in B$. With this ordering it is clear that $\mu$ is increasing.

We also have the following:

\begin{lemma}[Boole's inequality]
    Let $\mu$ be a measure on a Boolean algebra $B$. Then for any $x,y \in B$ we have
    %
    \begin{equation*}
        \mu(x \join y)
            \leq \mu(x) + \mu(y).
    \end{equation*}
\end{lemma}

\begin{proof}
    Notice that
    %
    \begin{equation*}
        (x \meet y') \meet y = x \meet (y' \meet y) = 0,
    \end{equation*}
    %
    and that
    %
    \begin{equation*}
        (x \meet y') \join y = (x \join y) \meet (y' \join y) = x \join y.
    \end{equation*}
    %
    It follows by additivity of $\mu$ that
    %
    \begin{equation*}
        \mu(x \join y)
            = \mu \bigl( (x \meet y') \join y \bigr)
            = \mu(x \meet y') + \mu(y)
            \leq \mu(x) + \mu(y),
    \end{equation*}
    %
    as desired.
\end{proof}


Next we equip Boolean algebras with a pseudometric. Let $B$ be a Boolean algebra. For $x,y \in B$ we define the \emph{symmetric difference} between $x$ and $y$ by
%
\begin{equation*}
    x \symdiff y
        = (x \meet y') \join (y \meet x').
\end{equation*}
%
If $\mu$ is a measure on $B$, define a map $\rho \colon B \times B \to [0,\infty)$ by
%
\begin{equation}
    \label{eq:Boolean-metric}
    \rho(x,y)
        = \mu(x \symdiff y)
\end{equation}
%
for $x,y \in B$.


\begin{lemma}
    The map $\rho$ is a pseudometric on $B$.
\end{lemma}

\begin{proof}
    We only need to prove the triangle inequality. To this end, let $x,y,z \in B$ and notice that
    %
    \begin{align*}
        x \meet z'
            &= (x \meet z) \meet (y' \join y) \\
            &= (x \meet z' \meet y') \join (x \meet z' \meet y) \\
            &\leq (x \meet y') \join (y \meet z').
    \end{align*}
    %
    Similarly we have $z \meet x' \leq (z \meet y') \join (y \meet x')$. It follows that
    %
    \begin{align*}
        x \symdiff z
            &= (x \meet z') \join (z \meet x') \\
            &\leq (x \meet y') \join (y \meet x') \join
                  (y \meet z') \join (z \meet y') \\
            &= (x \symdiff y) \join (y \symdiff z).
    \end{align*}
    %
    Now Boole's inequality implies that
    %
    \begin{equation*}
        \rho(x,z)
            = \mu(x \symdiff y)
            \leq \mu(x \symdiff y) + \mu(y \symdiff z)
            = \rho(x,y) + \rho(y,z),
    \end{equation*}
    %
    as desired.
\end{proof}

\begin{definition}[Metric Boolean algebras]
    A \emph{pseudometric Boolean algebra} is a tuple $(B,\mu,\rho)$, where $B$ is a Boolean algebra, $\mu$ is a measure on $B$, and $\rho$ is the pseudometric on $B$ given by \eqref{eq:Boolean-metric}.

    If $\rho$ is a metric, then $(B,\mu,\rho)$ is called a \emph{metric Boolean algebra}.
\end{definition}

Next we show that this pseudometric respects the lattice operations and complementation:

\begin{proposition}
    Let $(B, \mu, \rho)$ be a pseudometric Boolean algebra. Then the maps $x \mapsto x'$, $(x,y) \mapsto x \join y$, and $(x,y) \mapsto x \meet y$ are continuous.
\end{proposition}

\begin{proof}
    Let $x \in B$, and let $(x_n)_{n\in\naturals}$ be a sequence in $B$ that converges to $x$. Notice that $x_n' \symdiff x' = x_n \symdiff x$, so $\rho(x_n',x') = \rho(x_n,x)$. Hence the complementation map $x \mapsto x'$ is continuous.

    Let further $(y_n)_{n\in\naturals}$ be a sequence converging to a point $y \in B$. A short calculation shows that
    %
    \begin{align*}
        (x_n \join y_n) \symdiff (x \join y)
            &= (x_n \meet x' \meet y') \join
               (y_n \meet x' \meet y') \join
               (x \meet x_n' \meet y_n') \join
               (y \meet x_n' \meet y_n') \\
            &\leq (x_n \meet x') \join
            (x \meet x_n') \join
            (y_n \meet y') \join
            (y \meet y_n') \\
            &= (x_n \symdiff x) \join (y_n \symdiff y).
    \end{align*}
    %
    Thus Boole's inequality shows that
    %
    \begin{equation*}
        \rho(x_n \join y_n, x \join y)
            \leq \rho(x_n,x) + \rho(y_n,y),
    \end{equation*}
    %
    which implies continuity of the join map $(x,y) \mapsto x \join y$.

    Finally, continuity of the meet map $(x,y) \mapsto x \meet y$ follows since
    %
    \begin{equation*}
        x \meet y
            = \bigl( x' \join y' \bigr)',
    \end{equation*}
    %
    so it is a composition of continuous functions.
\end{proof}

It is well-known that any pseudometric space has a completion, i.e. can be isometrically embedded as a dense subset of a complete pseudometric space. See for instance Willard Corollary~24.5. A natural question is then: If $(B,\rho)$ is a pseudometric Boolean algebra with metric completion $(\altoverline{B}, \altoverline{\rho})$, does $\altoverline{B}$ also carry the structure of a Boolean algebra? (Yes, obvious since operations are continuous.)

\begin{lemma}
    \label{thm:monotonic-sequence-Cauchy}
    Let $(B,\rho)$ be a pseudometric Boolean algebra. Every monotonic sequence in $B$ is a Cauchy sequence.
\end{lemma}

\begin{proof}
    Let $(x_n)_{n\in\naturals}$ be an increasing sequence in $B$. Then since $x_n \leq 1$ we also have $\mu(x_n) \leq \mu(1) < \infty$ for all $n \in \naturals$. Thus the sequence $(\mu(x_n))$ is a bounded increasing sequence in $\setR$, hence it converges to some $\alpha \geq 0$. For $\epsilon > 0$ there is an $N \in \naturals$ such that $\mu(x_n) \in (\alpha - \epsilon, \alpha]$ for all $n \geq N$.
    
    If then $m,n \geq N$ with $m \leq n$, then $x_m \leq x_n$ and so
    %
    \begin{equation*}
        x_m \meet x_n'
            \leq x_n \meet x_n'
            = 0.
    \end{equation*}
    %
    Hence $x_m \symdiff x_n = x_n \meet x_m'$, so it follows that
    %
    \begin{equation*}
        \rho(x_m,x_n)
            = \mu(x_n \meet x_m')
            = \mu(x_n) - \mu(x_m)
            < \epsilon.
    \end{equation*}
    %
    Thus $(x_n)$ is indeed a Cauchy sequence. The case where $(x_n)$ is decreasing is similar.
\end{proof}


\begin{proposition}
    Let $(B,\rho)$ be a pseudometric Boolean algebra with completion $(\altoverline{B}, \altoverline{\rho})$. Any $(x_n)_{n\in\naturals}$ is a sequence in $B$ has a join in $\altoverline{B}$, and
    %
    \begin{equation*}
        \bigjoin_{n\in\naturals} x_n
            = \lim_{n\to\infty} \bigjoin_{i \leq n} x_i.
    \end{equation*}
    %
    Similarly for meets.
\end{proposition}
%
[Something about limits not being unique... Is this a problem, since meets and joins are unique?]

\begin{proof}
    The sequence $( \bigjoin_{i \leq n} x_i )_{n\in\naturals}$ is increasing, so by \cref{thm:monotonic-sequence-Cauchy} it has a limit $s \in \altoverline{B}$. For $k \in \naturals$ and $n \geq k$ we have $x_k \leq \bigjoin_{i \leq n} x_i$, i.e.
    %
    \begin{equation*}
        x_k \join \bigjoin_{i \leq n} x_i
            = \bigjoin_{i \leq n} x_i.
    \end{equation*}
    %
    Taking the limit as $n \to \infty$, continuity of (binary) joins implies that $x_k \join s = s$, or $x_k \leq s$. Thus $s$ is an upper bound of the sequence $(x_n)$.

    On the other hand, if $t \in \altoverline{B}$ is an upper bound of $(x_n)$, then $x_k \leq t$. We have just seen that taking limits preserves inequalities, so this implies that $s \leq t$ as desired.

    The corresponding result for meets follows similarly, or from the fact that complementation is continuous.
\end{proof}


\chapter{Event spaces as Boolean algebras}

If the probability of an event is supposed to be a measure of how often this event occurs, then it seems reasonable to assume that we are, in principle, able to distinguish when this event occurs. For example, rolling a six-sided die the state of affairs \textquote{the result of the die roll is three} is an event, since we can determine the outcome of the roll just by looking at the die. To take another example, throwing a ball the state of affairs \textquote{the ball was thrown more than 50 metres} is also an event: That is, we can determine whether or not the length of the throw was strictly greater than 50 metres.

One might take a different view: One might agree that it is possible to \emph{affirm} that the length of the throw, measured in metres, lies in the interval $(50,\infty)$. If the length $L$ does in fact lie in the above interval, we can simply take a ruler whose subdivisions are smaller than $L - 50$ in metres. However, one might disagree that it is possible to \emph{refute} that $L \in (50, \infty)$. For if $L$ is exactly $50$ metres, then since any measurement of $L$ carries some error, it is in practice impossible to determine whether $L$ is $50$ (or slightly smatter), or whether it is slightly larger than $50$. We will not pursue this line further but refer the reader to \textcite{vickers1989} for more on this \emph{logic of affirmative assertions}.

To be precise, after performing the relevant random experiment, we will assume that we are always able to decide whether or not the event has occured or not. In particular, if $E$ is an event, then the state of affairs \textquote{$E$ does not obtain} is also an event, denoted $E'$: If $E$ obtains, then $E'$ does not. And conversely, if $E$ does not obtain, then $E'$ does obtain. We call $E'$ the \emph{complement of} or the \emph{complementary event to $E$}.\footnote{In contrast, in the logic of affirmative assertions we do not allow complementation (i.e. negation). Hence it may not be surprising that this logic ends up being closely tied to topology.}

Next consider two events $E_1$ and $E_2$. Since we are able to decide whether each of them have obtained, the same is true for the event \textquote{both $E_1$ and $E_2$ have obtained} and the event \textquote{at least one of $E_1$ and $E_2$ has obtained}. The first is called the \emph{conjunction} of $E_1$ and $E_2$ and is denoted $E_1 \meet E_2$, and the second is the \emph{disjunction} $E_1 \join E_2$ of $E_1$ and $E_2$.

Finally it seems natural to allow an \textquote{empty event} $0$ which never occurs, as well as a \textquote{universal event} $1$ that always occurs.

With these operations in place, we claim that this logic of events constitute a \emph{Boolean algebra}. We recall the definition of such a structure:

\begin{definition}[Boolean algebras]
    A \emph{Boolean algebra} is a structure $\langle B; \join, \meet, ', 0, 1 \rangle$ such that
    %
    \begin{enumdef}
        \item $\langle B; \join, \meet \rangle$ is a distributive lattice,
        \item $a \join 0 = a$ and $a \meet 1$ for all $a \in B$, and
        \item $a \join a' = 1$ and $a \meet a' = 0$ for all $a \in B$.
    \end{enumdef}
\end{definition}
%
We leave it to the reader to reflect on whether or not Boolean algebras offer a reasonable way to formalise our intuition of events (in the opinion of the author, they do). This leads naturally to the following definition:

\begin{definition}[Generalised abstract probability spaces]
    A \emph{generalised abstract probability space} is a pair $(\calF, \P)$, where $\calF$ is a set and $\P \colon \calF \to [0,1]$, such that
    %
    \begin{enumdef}
        \item $\langle \calF; \join, \meet, ', 0, 1 \rangle$ is a Boolean algebra,
        \item $\P(1) = 1$, and
        \item $E \meet F = 0$ implies $\P(E \join F) = \P(E) + \P(F)$ for all $E,F \in \calF$.
    \end{enumdef}
\end{definition}
%
In [ref] we consider more restrictive abstract probability spaces, hence the adjective \textquote{generalised}. We refer to the property (iii) as \emph{(finite) additivity} of $\P$. If $E$ and $F$ are events such that $E \meet F = 0$, then we call $E$ and $F$ \emph{disjoint} or \emph{incompatible}.

We define an ordering on events in the usual way: If $E$ and $F$ are events such that $E \join F = F$, then we write $E \leq F$ and say that \emph{$E$ implies $F$}. The intuition is that, if $F$ obtains just when \emph{either} $E$ or $F$ obtains, then there is no way for $E$ to obtain without $F$ also doing so.


\begin{lemma}
    Let $(\calF, \P)$ be a generalised abstract probability space.
    %
    \begin{enumlem}
        \item $\P(0) = 0$.
        \item $\P$ is increasing, i.e. $E \leq F$ implies $\P(E) \leq \P(F)$ for all $E,F \in \calF$.
    \end{enumlem}
\end{lemma}

\begin{proof}
    The first claim follows since $0 \join 0 = 0 \meet 0 = 0$, so by additivity of $\P$,
    %
    \begin{equation*}
        \P(0)
            = \P(0 \join 0)
            = \P(0) + \P(0),
    \end{equation*}
    %
    hence $\P(0) = 0$.

    To prove the second claim, first notice that if $E \leq F$ then $F = E \join (F \meet E')$. But $E \meet (F \meet E') = 0$, so
    %
    \begin{equation*}
        \P(F)
            = \P \bigl( E \join (F \meet E') \bigr)
            = \P(E) + \P(F \meet E')
            \geq \P(E),
    \end{equation*}
    %
    as claimed.
\end{proof}


\chapter[Abstract sigma-algebras and continuity][Abstract $\sigma$-algebras and continuity]{Abstract $\sigma$-algebras and continuity}

Of course, the map $\P$ above is supposed to be a probability measure. But measures are \emph{countably} additive, not just finitely so. It is however difficult to justify this on conceptual groups. In fact, according to Kolmogorov himself:
%
\blockquote[\cite{kolmogorov1956}]{%
    Since the new axiom \textins{countable additivity} is essential for infinite fields of probability only, it is almost impossible to elucidate its empirical meaning. \textelp{} For, in describing any observable random process we can obtain only finite fields of probability. Infinite fields of probability occur only as idealized models of real random processes. \emph{We limit ourselves, arbitrarily, to only those models which satisfy Axiom VI} \textins{countable additivity}. This limitation has been found expedient in researches of the most diverse sort.%
}
%
And furthermore:
%
\blockquote[\cite{kolmogorov1995}]{%
    \textins*{S}omewhat more complicated problems require, if the theory is to be simple and tractable, that probability be subject to the \emph{axiom of denumerable additivity}. However, the justification of that axiom remains purely empirical, in that we have not yet encountered any interesting problem for which we have not been able to construct a probability field conforming to the axiom in question.%
}
%
\textcite{kolmogorov1956} introduces the axiom in the following form:

\begin{definition}[Axiom of continuity]
    A generalised abstract probability space $(\calF, \P)$ is said to satisfy the \emph{axiom of continuity} if it has the following property: If $(E_n)_{n\in\naturals}$ is a decreasing sequence of events in $\calF$ such that $\bigmeet_{n\in\naturals} E_n$ exists and equals $0$, then $\lim_{n\to\infty} \P(E_n) = 0$.
\end{definition}
%
This axiom implies that $\P$ is countably additive when the join of a sequence of elements in $\calF$ also lies in $\calF$. To prove this fact we need a lemma, which is obvious if $\calF$ is a set algebra:

\begin{lemma}
    \label{thm:refine_join}
    Let $\langle B; \join, \meet, ', 0, 1 \rangle$ be a Boolean algebra. Let $(a_i)_{i \in I}$ be a collection of elements in $B$ such that $a_i \meet a_j = 0$ when $i \neq j$. If $\bigjoin_{i \in I} a_i \in B$, then $\bigjoin_{i \in J} a_i \in B$ for any $J \subseteq I$ such that $I \setminus J$ is finite (i.e. $J$ is cofinite).
\end{lemma}

\begin{proof}
    Let $(a_i)_{i \in I}$ be such a collection of elements, and let $J \subseteq I$ be cofinite. It suffices to prove the lemma in the case $I \setminus J = \{i_0\}$, since the general case then follows by induction. We claim that
    %
    \begin{equation*}
        \bigjoin_{i \in J} a_i
            = a_{i_0}' \meet \bigjoin_{i \in I} a_i.
    \end{equation*}
    %
    Let $j \in J$ and notice that, since $a_{i_0} \meet a_j = 0$,
    %
    \begin{equation*}
        a_{i_0}' \meet a_j
            = (a_{i_0}' \meet a_j) \join (a_{i_0} \meet a_j)
            = (a_{i_0}' \join a_{i_0}) \meet a_j
            = 1 \meet a_j
            = a_j.
    \end{equation*}
    %
    Now because $a_j \leq \bigjoin_{i \in I} a_i$ we get
    %
    \begin{equation*}
        a_j
            = a_{i_0}' \meet a_j
            \leq a_{i_0}' \meet \bigjoin_{i \in I} a_i.
    \end{equation*}
    %
    Conversely, suppose that $a_j \leq s$ for all $j \in J$. Then $a_i \leq a_{i_0} \join s$ for all $i \in I$, so
    %
    \begin{equation*}
        \bigjoin_{i \in I} a_i
            \leq a_{i_0} \join s.
    \end{equation*}
    %
    It follows that
    %
    \begin{equation*}
        a_{i_0}' \meet \bigjoin_{i \in I} a_i
            \leq a_{i_0}' \meet (a_{i_0} \join s)
            = 0 \join (a_{i_0} \meet s)
            \leq s,
    \end{equation*}
    %
    as desired.
\end{proof}


\begin{proposition}[The generalised addition theorem]
    Let $(\calF, \P)$ be a generalised abstract probability space satisfying the axiom of continuity. If $(E_n)_{n\in\naturals}$ be a sequence of pairwise disjoint events in $\calF$ such that $E = \bigjoin_{n\in\naturals} E_n \in \calF$, then
    %
    \begin{equation*}
        \P(E)
            = \sum_{n=1}^\infty \P(E_n).
    \end{equation*}
\end{proposition}

\begin{proof}
    By \cref{thm:refine_join}, $R_n = \bigjoin_{i > n} E_i$ exists in $\calF$, and we claim that $\bigmeet_{n\in\naturals} R_n = 0$. Let $L \in \calF$ be a lower bound of $R_n$ for $n \in \naturals$. Then for $n \in \naturals$ we have
    %
    \begin{equation*}
        L \join \bigjoin_{i > n} E_i
            = \bigjoin_{i > n} E_i,
    \end{equation*}
    %
    and taking the meet of each side with $E_n'$ yields $L \meet E_n' = 0$. Hence $\bigjoin_{n\in\naturals} L \meet E_n = 0$. Now notice that $\bigmeet_{n\in\naturals} E_n = 0$, since $E_n \meet E_m = 0$ when $n \neq m$. It follows that $\bigjoin_{n\in\naturals} E_n' = 1$, and so
    %
    \begin{equation*}
        L
            = L \meet \bigjoin_{n\in\naturals} E_n'
            = \bigjoin_{n\in\naturals} L \meet E_n'
            = 0.
    \end{equation*}
    %
    Thus $\bigmeet_{n\in\naturals} R_n = 0$ as claimed. It now follows from finite additivity of $\P$ and the axiom of continuity that
    %
    \begin{equation*}
        \P(E)
            = \sum_{i=1}^n \P(E_i) + \P(R_n)
            \to \sum_{i=1}^\infty \P(E_i)
    \end{equation*}
    %
    as $n \to \infty$ as desired.
\end{proof}

Of course, we are not always guaranteed that a generalised abstract probability space is closed under countable joins.


\begin{definition}[Abstract $\sigma$-algebra]
    An \emph{abstract $\sigma$-algebra} is a Boolean algebra $\langle B; \join, \meet, ', 0, 1 \rangle$ with countable joins. That is, if $(a_n)_{n\in\naturals}$ is a sequence of elements in $B$, then their join $\bigjoin_{n\in\naturals} a_n$ exists.
\end{definition}
%
If the join $\bigjoin_{n\in\naturals} a_n$ exists, then it follows by taking complements that the meet $\bigmeet_{n\in\naturals} a_n$ also exists.


\nocite{*}

\printbibliography


\end{document}