% Document setup
\documentclass[article, a4paper, 11pt, oneside]{memoir}
\usepackage[utf8]{inputenc}
\usepackage[T1]{fontenc}
\usepackage[UKenglish]{babel}

% Document info
\newcommand\doctitle{Foundations of probability theory}
\newcommand\docauthor{Danny Nygård Hansen}

% Formatting and layout
\usepackage[autostyle]{csquotes}
\renewcommand{\mktextelp}{(\textellipsis\unkern)}
\usepackage[final]{microtype}
\usepackage{xcolor}
\frenchspacing
\usepackage{latex-sty/articlepagestyle}
\usepackage{latex-sty/articlesectionstyle}

% Fonts
\usepackage[largesmallcaps,partialup]{kpfonts}
\DeclareSymbolFontAlphabet{\mathrm}{operators} % https://tex.stackexchange.com/questions/40874/kpfonts-siunitx-and-math-alphabets
\linespread{1.06}
\let\mathfrak\undefined
\usepackage{eufrak}
\usepackage{inconsolata}
\usepackage{amssymb}

% Hyperlinks
\usepackage{hyperref}
\definecolor{linkcolor}{HTML}{4f4fa3}
\hypersetup{%
	pdftitle=\doctitle,
	pdfauthor=\docauthor,
	colorlinks,
	linkcolor=linkcolor,
	citecolor=linkcolor,
	urlcolor=linkcolor,
	bookmarksnumbered=true
}

% Equation numbering
\numberwithin{equation}{chapter}

% Footnotes
\footmarkstyle{\textsuperscript{#1}\hspace{0.25em}}

% Mathematics
\usepackage{latex-sty/basicmathcommands}
\usepackage{latex-sty/framedtheorems}
\usepackage{latex-sty/probabilitycommands}
\usepackage{tikz-cd}
\usetikzlibrary{babel}

% Lists
\usepackage{enumitem}
\setenumerate[0]{label=\normalfont(\arabic*)}

% Bibliography
\usepackage[backend=biber, style=authoryear, maxcitenames=2, useprefix]{biblatex}
\addbibresource{references.bib}

% Title
\title{\doctitle}
\author{\docauthor}


% Section style -- add to section style .sty?
\setsubsecheadstyle{\normalfont\itshape}


% Preimage -- to be added to mathcommands .sty
\newcommand{\preim}{^{-1}}


\newcommand{\calN}{\mathcal{N}}
\DeclarePairedDelimiter{\nhoodfilteraux}{(}{)}
% \newcommand{\nhoodfilter}[1]{\calN\nhoodfilteraux{#1}}
\newcommand{\nhoodfilter}[1]{\calN_{#1}}


\newcommand{\calU}{\mathcal{U}}
\newcommand{\calV}{\mathcal{V}}
\newcommand{\calW}{\mathcal{W}}
\newcommand{\calT}{\mathcal{T}}
\newcommand{\calB}{\mathcal{B}}
\newcommand{\calE}{\mathcal{E}}
\newcommand{\calF}{\mathcal{F}}
\newcommand{\calA}{\mathcal{A}}
\newcommand{\calD}{\mathcal{D}}
\newcommand{\calS}{\mathcal{S}}

\newcommand{\borel}[1]{\calB(#1)}
\DeclareMathOperator{\supp}{supp}

\let\oldP\P
\renewcommand{\P}{\mathbb{P}}

% Add to basicmathcommands.sty
\DeclarePairedDelimiter{\card}{\lvert}{\rvert}
\renewcommand{\symdiff}{\mathbin{\triangle}}


\begin{document}

\maketitle

\chapter{Introduction}

In the usual measure-theoretical formulation of probability theory, the following result is a corollary of the Law of Large Numbers:

\begin{theorem}[The frequency interpretation of probability]
    Let $\rvar{X}, \rvar{X}_1, \rvar{X}_2, \ldots$ be i.i.d. real-valued random variables on a probability space $(\Omega, \calF, \P)$. For every $B \in \borel{\reals}$ we have
    %
    \begin{equation*}
        \P(\rvar{X} \in B)
            = \lim_{n\to\infty} \frac{
                \card{ \set{j \in \{1, \ldots, n\} }{ \rvar{X}_j(\omega) \in B } }
            }{
                n
            }
    \end{equation*}
    %
    for $\P$-almost all $\omega \in \Omega$.
\end{theorem}

\begin{proof}
    Thorbjørnsen Korollar~13.6.2. [Maybe reproduce the proof given LLN for completeness?]
\end{proof}
%
That is, given a sequence $(\rvar{X}_n)_{n\in\naturals}$, and one extra $\rvar{X}$, of i.i.d. random variables, the probability that $\rvar{X}$ lies in some Borel set $B$ can be thought of as the proportion of the $\rvar{X}_n$ that lie in $B$, as $n$ tends to infinity. In other words, probability is a measure of the \emph{frequency} with which an outcome of a random experiment obtains, if we repeat the experiment many times.

Whether or not this is the correct interpretation of probability as it occurs in the natural world we will not discuss here. Nonetheless the above result is an uncontroversial consequence of the theory, and it certainly aligns with our intuitive understanding of probability.

In this note we turn this result on its head and attempt to use it to motivate the formalisation of probability theory in terms of measure spaces. As we shall see, this is not entirely successful and will require some leaps that are not entirely justified by our conceptual grasp of probability.


\chapter{Preliminaries}


\section{Boolean algebras}

We begin by reviewing some of the purely algebraic properties of Boolean algebras.

\begin{definition}[Boolean algebras]
    \label{def:Boolean-algebra}
    A \emph{Boolean algebra} is a structure $\langle B; \join, \meet, ', 0, 1 \rangle$ such that
    %
    \begin{enumdef}
        \item $\langle B; \join, \meet \rangle$ is a distributive lattice,
        \item $0$ and $1$ are elements of $B$ such that $x \join 0 = x$ and $x \meet 1$ for all $x \in B$, and
        \item $'$ is a unary operation such that $x \join x' = 1$ and $x \meet x' = 0$ for all $x \in B$.
    \end{enumdef}
\end{definition}
%
The binary operations $\join$ and $\meet$ are called \emph{join} and \emph{meet}, respectively. For $x \in B$ the element $x'$ is called the \emph{complement} of $x$. In a general bounded lattice $L$, an element $y \in L$ such that $x \join y = 1$ and $x \meet y = 0$ is called a complement of $x \in L$. If $L$ is distributive, complements are unique. Recall also that the lattice structure on $B$ induces a partial order $\leq$ such that $x \leq y$ if and only if $x \join y = y$ for $x,y \in B$.

Let $B$ be a Boolean algebra. For $x,y \in B$ we define the \emph{symmetric difference} between $x$ and $y$ by
%
\begin{equation*}
    x \symdiff y
        = (x \meet y') \join (y \meet x').
\end{equation*}
%
If $x \symdiff y = 0$, then it is easy to show that $x = y$.

Before proceeding we note the following technical result that we shall need later:

\begin{lemma}
    \label{thm:refine_join}
    Let $\langle B; \join, \meet, ', 0, 1 \rangle$ be a Boolean algebra. Let $(x_i)_{i \in I}$ be a collection of elements in $B$ such that $x_i \meet x_j = 0$ when $i \neq j$. If $\bigjoin_{i \in I} x_i \in B$, then $\bigjoin_{i \in J} x_i \in B$ for any cofinite\footnotemark{} $J \subseteq I$.
\end{lemma}
\footnotetext{Recall that a subset $J$ of a set $I$ is called \emph{cofinite} if the complement $I \setminus J$ is finite.}

\begin{proof}
    Let $(x_i)_{i \in I}$ be such a collection of elements, and let $J \subseteq I$ be cofinite. It suffices to prove the lemma in the case $I \setminus J = \{i_0\}$, since the general case then follows by induction. We claim that
    %
    \begin{equation*}
        \bigjoin_{i \in J} x_i
            = x_{i_0}' \meet \bigjoin_{i \in I} x_i.
    \end{equation*}
    %
    Let $j \in J$ and notice that, since $x_{i_0} \meet x_j = 0$,
    %
    \begin{equation*}
        x_{i_0}' \meet x_j
            = (x_{i_0}' \meet x_j) \join (x_{i_0} \meet x_j)
            = (x_{i_0}' \join x_{i_0}) \meet x_j
            = 1 \meet x_j
            = x_j.
    \end{equation*}
    %
    Now because $x_j \leq \bigjoin_{i \in I} x_i$ we get
    %
    \begin{equation*}
        x_j
            = x_{i_0}' \meet x_j
            \leq x_{i_0}' \meet \bigjoin_{i \in I} x_i.
    \end{equation*}
    %
    Conversely, suppose that $x_j \leq s$ for all $j \in J$. Then $x_i \leq x_{i_0} \join s$ for all $i \in I$, so
    %
    \begin{equation*}
        \bigjoin_{i \in I} x_i
            \leq x_{i_0} \join s.
    \end{equation*}
    %
    It follows that
    %
    \begin{equation*}
        x_{i_0}' \meet \bigjoin_{i \in I} x_i
            \leq x_{i_0}' \meet (x_{i_0} \join s)
            = 0 \join (x_{i_0} \meet s)
            \leq s,
    \end{equation*}
    %
    as desired.
\end{proof}



\section{Abstract measure spaces}

\begin{definition}[Generalised abstract measure spaces]
    A \emph{measure} on a Boolean algebra $B$ is a map $\mu \colon B \to [0,\infty)$ such that $x \meet y = 0$ implies
    %
    \begin{equation}
        \label{eq:finite-additivity}
        \mu(x \join y)
            = \mu(x) + \mu(y)
    \end{equation}
    %
    for all $x,y \in B$. If $\mu$ is a measure on a Boolean algebra $B$, then we call the pair $(B,\mu)$ a \emph{generalised abstract measure space}. If $x \neq 0$ implies that $\mu(x) > 0$, then $\mu$ is called \emph{positive definite}. If $\mu(1) = 1$, then we call $\mu$ a \emph{probability measure}.
\end{definition}
%
It is clear that $\mu(\emptyset) = 0$ and that $\mu$ is increasing. It follows that $\mu(x) \leq \mu(1)$ for all $x \in B$. Notice that we require that $\mu$ is finite, but this is no restriction since we are ultimately interested in the case where $\mu$ is a probability measure.

The property \cref{eq:finite-additivity} is called \emph{(finite) additivity} of $\mu$. We will later define more restrictive structures and measures upon them, hence the adjective \enquote{generalised}.

\begin{proposition}[Boole's inequality]
    Let $(B,\mu)$ be a generalised abstract measure space $B$. Then for any $x,y \in B$ we have
    %
    \begin{equation*}
        \mu(x \join y)
            \leq \mu(x) + \mu(y).
    \end{equation*}
\end{proposition}

\begin{proof}
    Notice that
    %
    \begin{equation*}
        (x \meet y') \meet y = x \meet (y' \meet y) = 0,
    \end{equation*}
    %
    and that
    %
    \begin{equation*}
        (x \meet y') \join y = (x \join y) \meet (y' \join y) = x \join y.
    \end{equation*}
    %
    It follows by additivity of $\mu$ that
    %
    \begin{equation*}
        \mu(x \join y)
            = \mu \bigl( (x \meet y') \join y \bigr)
            = \mu(x \meet y') + \mu(y)
            \leq \mu(x) + \mu(y),
    \end{equation*}
    %
    as desired.
\end{proof}


\begin{definition}[Metric Boolean algebras]
    A \emph{pseudometric Boolean algebra} is a tuple $(B,\rho)$, where $B$ is a Boolean algebra and $\rho$ is a pseudometric on $B$ such that the maps $x \mapsto x'$, $(x,y) \mapsto x \join y$, and $(x,y) \mapsto x \meet y$ are continuous.

    If $\rho$ is a metric, then $(B,\rho)$ is called a \emph{metric Boolean algebra}.
\end{definition}

Next we equip generalised abstract measure spaces with a canonical pseudometric. If $(B,\mu)$ is a generalised abstract measure space, define a map $\rho_\mu \colon B \times B \to [0,\infty)$ by
%
\begin{equation}
    \label{eq:Boolean-metric}
    \rho_\mu(x,y)
        = \mu(x \symdiff y)
\end{equation}
%
for $x,y \in B$. The next proposition shows that $\rho_\mu$ is in fact a pseudometric. We will always equip a generalised abstract measure space with this pseudometric.

\begin{proposition}
    Given a generalised abstract measure space $(B,\mu)$, the map $\rho_\mu$ defined in \eqref{eq:Boolean-metric} makes $(B,\rho_\mu)$ into a pseudometric Boolean algebra. Furthermore, $\rho_\mu$ is a metric if and only if $\mu$ is positive definite.
\end{proposition}

% Add to framedtheorems.sty
\newcommand{\mylistlabelfont}[1]{{\normalfont\color{linkcolor}\textit{#1}:}}
\newlist{proofsec}{description}{1}
\setlist[proofsec]{leftmargin=0pt, parsep=0pt, listparindent=\parindent, font=\mylistlabelfont}

\begin{proof}
\begin{proofsec}
    \item[$\rho_\mu$ is a pseudometric]
    We only need to prove the triangle inequality. To this end, let $x,y,z \in B$ and notice that
    %
    \begin{align*}
        x \meet z'
            &= (x \meet z) \meet (y' \join y) \\
            &= (x \meet z' \meet y') \join (x \meet z' \meet y) \\
            &\leq (x \meet y') \join (y \meet z').
    \end{align*}
    %
    Similarly we have $z \meet x' \leq (z \meet y') \join (y \meet x')$. It follows that
    %
    \begin{align*}
        x \symdiff z
            &= (x \meet z') \join (z \meet x') \\
            &\leq (x \meet y') \join (y \meet x') \join
                  (y \meet z') \join (z \meet y') \\
            &= (x \symdiff y) \join (y \symdiff z).
    \end{align*}
    %
    Now Boole's inequality implies that
    %
    \begin{equation*}
        \rho_\mu(x,z)
            = \mu(x \symdiff y)
            \leq \mu(x \symdiff y) + \mu(y \symdiff z)
            = \rho_\mu(x,y) + \rho_\mu(y,z),
    \end{equation*}
    %
    as desired.

    \item[Continuity of lattice operations]
    Let $x \in B$, and let $(x_n)_{n\in\naturals}$ be a sequence in $B$ that converges to $x$. Notice that $x_n' \symdiff x' = x_n \symdiff x$, so $\rho_\mu(x_n',x') = \rho_\mu(x_n,x)$. Hence the complementation map $x \mapsto x'$ is continuous.

    Let further $(y_n)_{n\in\naturals}$ be a sequence converging to a point $y \in B$. A short calculation shows that
    %
    \begin{align*}
        (x_n \join y_n) \symdiff (x \join y)
            &= (x_n \meet x' \meet y') \join
               (y_n \meet x' \meet y') \join
               (x \meet x_n' \meet y_n') \join
               (y \meet x_n' \meet y_n') \\
            &\leq (x_n \meet x') \join
            (x \meet x_n') \join
            (y_n \meet y') \join
            (y \meet y_n') \\
            &= (x_n \symdiff x) \join (y_n \symdiff y).
    \end{align*}
    %
    Thus Boole's inequality shows that
    %
    \begin{equation}
        \label{eq:join-continuous}
        \rho_\mu(x_n \join y_n, x \join y)
            \leq \rho_\mu(x_n,x) + \rho_\mu(y_n,y),
    \end{equation}
    %
    which implies continuity of the join map $(x,y) \mapsto x \join y$.

    Finally, continuity of the meet map $(x,y) \mapsto x \meet y$ follows since
    %
    \begin{equation*}
        x \meet y
            = \bigl( x' \join y' \bigr)',
    \end{equation*}
    %
    so it is a composition of continuous functions.

    \item[Positive definiteness]
    The last claim follows directly from the fact that $x \symdiff y = 0$ if and only if $x = y$ for all $x,y \in B$.
\end{proofsec}
\end{proof}

\begin{remark}
    \label{rem:convergence-of-measure}
    Notice that the measure $\mu$ can be written in terms of $\rho_\mu$, since $\mu(x) = \rho_\mu(x,0)$. Furthermore, since (pseudo)metrics are continuous, it follows that $\mu(x_n) \to \mu(x)$ whenever $x_n \to x$ in $B$.
\end{remark}


It is well-known that any (pseudo)metric space has a completion, i.e. can be isometrically embedded as a dense subset of a complete (pseudo)metric space. See for instance Willard Corollary~24.5. A natural question is then: If $(B,\rho)$ is a (pseudo)metric Boolean algebra with metric completion $(\altoverline{B}, \altoverline{\rho})$, does $\altoverline{B}$ also carry the structure of a Boolean algebra?

This is indeed the case, and we sketch the construction: Let $x,y \in \altoverline{B}$, and let $(x_n)$ and $(y_n)$ be sequences in $B$ that converge to $x$ and $y$, respectively. Then these are Cauchy sequences in $B$, and the calculation leading to \eqref{eq:join-continuous} show that $(x_n \join y_n)$ is also a Cauchy sequence. Thus it converges to some element of $\altoverline{B}$. Denote it $x \join y$. We define $x'$ and $x \meet y$ similarly. It is easy to check that these operations satisfy the conditions in \cref{def:Boolean-algebra}. Furthermore, the completion $\altoverline{\rho}$ of the pseudometric $\rho$ makes $(\altoverline{B}, \altoverline{\rho})$ into a pseudometric Boolean algebra.

This takes care of the metric structure. The next proposition shows that we can also extend the measure on a generalised abstract measure space to its completion.

\begin{proposition}
    Let $(B,\mu)$ be a generalised abstract measure space, and let $(\altoverline{B}, \altoverline{\rho}_\mu)$ be the completion of $(B,\rho_\mu)$. Define a map $\altoverline{\mu} \colon \altoverline{B} \to [0,\infty)$ by
    %
    \begin{equation}
        \label{eq:mu-overline-def}
        \altoverline{\mu}(x) = \lim_{n\to\infty} \mu(x_n),
    \end{equation}
    %
    where $(x_n)_{n\in\naturals}$ is any sequence in $B$ that converges to $x$. Then $\altoverline{\mu}$ is a well-defined measure on $\altoverline{B}$. The generalised abstract measure space $(\altoverline{B}, \altoverline{\mu})$ is called the \emph{completion} of $(B,\mu)$.
\end{proposition}

\begin{proof}
    First notice that for any $x \in \altoverline{B}$ there does in fact exist a sequence in $B$ converging to $x$. If $(x_n)_{n\in\naturals}$ is such a sequence, it is a Cauchy sequence in $B$, and the reverse triangle inequality shows that $(\mu(x_n))$ is a Cauchy sequence in $\reals$, hence convergent. Thus the limit on the right-hand side of \eqref{eq:mu-overline-def} exists.
    
    Now let $(y_n)$ be another sequence in $B$ that approximates $x$. Another application of the reverse triangle inequality then shows that
    %
    \begin{equation*}
        \abs{ \mu(x_n) - \mu(y_n) }
            \leq \rho_\mu(x_n,y_n)
            \leq \rho_\mu(x_n,x) + \rho_\mu(y_n,x)
            \to 0.
    \end{equation*}
    %
    Hence $\mu(x_n)$ and $\mu(y_n)$ converge to the same value, and thus $\altoverline{\mu}$ is well-defined.

    Next we show that $\altoverline{\mu}$ is finitely additive. Let $x,y \in \altoverline{B}$ with $x \meet y = 0$ and choose approximating sequences $(x_n)$ and $(y_n)$ in $B$. Then
    %
    \begin{equation*}
        (x_n \join y_n) \meet (x_n \meet y_n)'
            = \bigl( x_n \meet (x_n \meet y_n)' \bigr)
              \join \bigl( y_n \meet (x_n \meet y_n)' \bigr)
    \end{equation*}
    %
    is the join of disjoint elements of $B$, so
    %
    \begin{equation*}
        \mu \bigl( (x_n \join y_n) \meet (x_n \meet y_n)' \bigr)
            = \mu \bigl( x_n \meet (x_n \meet y_n)' \bigr)
              + \mu \bigl( y_n \meet (x_n \meet y_n)' \bigr).
    \end{equation*}
    %
    By continuity of the lattice operations we have $(x_n \meet y_n)' \to 1$, so the three elements given as arguments to $\mu$ above are elements in approximating sequences for $x \join y$, $x$ and $y$ respectively. By definition of $\altoverline{\mu}$ it follows that
    %
    \begin{equation*}
        \altoverline{\mu}(x \join y)
            = \altoverline{\mu}(x) + \altoverline{\mu}(y)
    \end{equation*}
    %
    as desired.
\end{proof}


\begin{lemma}
    \label{thm:monotonic-sequence-Cauchy}
    Let $(B,\mu)$ be a generalised abstract measure space. Every monotonic sequence in $B$ is a Cauchy sequence.
\end{lemma}

\begin{proof}
    Let $(x_n)_{n\in\naturals}$ be an increasing sequence in $B$. Then since $x_n \leq 1$ we also have $\mu(x_n) \leq \mu(1) < \infty$ for all $n \in \naturals$. Thus the sequence $(\mu(x_n))$ is a bounded increasing sequence in $\setR$, hence it converges to some $\alpha \geq 0$. For $\epsilon > 0$ there is an $N \in \naturals$ such that $\mu(x_n) \in (\alpha - \epsilon, \alpha]$ for all $n \geq N$.
    
    If then $m,n \geq N$ with $m \leq n$, then $x_m \leq x_n$ and so
    %
    \begin{equation*}
        x_m \meet x_n'
            \leq x_n \meet x_n'
            = 0.
    \end{equation*}
    %
    Hence $x_m \symdiff x_n = x_n \meet x_m'$, so it follows that
    %
    \begin{equation*}
        \rho_\mu(x_m,x_n)
            = \mu(x_n \meet x_m')
            = \mu(x_n) - \mu(x_m)
            < \epsilon.
    \end{equation*}
    %
    Thus $(x_n)$ is indeed a Cauchy sequence. The case where $(x_n)$ is decreasing is similar.
\end{proof}


\begin{proposition}
    \label{thm:existence-of-joins}
    Let $(B,\rho)$ be a metric Boolean algebra with completion $(\altoverline{B}, \altoverline{\rho})$. Any sequence $(x_n)_{n\in\naturals}$ in $B$ has a join in $\altoverline{B}$, and
    %
    \begin{equation*}
        \bigjoin_{n\in\naturals} x_n
            = \lim_{n\to\infty} \bigjoin_{i \leq n} x_i.
    \end{equation*}
    %
    Similarly for meets.
\end{proposition}
%
As far as I know, it is not possible to generalise this result to the case where $\rho$ is only a pseudometric. [We also need the measure, don't we? For the lemma? Kolmogorov only looks at this when there is a positive definite measure.]

\begin{proof}
    The sequence $( \bigjoin_{i \leq n} x_i )_{n\in\naturals}$ is increasing, so by \cref{thm:monotonic-sequence-Cauchy} it has a limit $s \in \altoverline{B}$. For $k \in \naturals$ and $n \geq k$ we have $x_k \leq \bigjoin_{i \leq n} x_i$, i.e.
    %
    \begin{equation*}
        x_k \join \bigjoin_{i \leq n} x_i
            = \bigjoin_{i \leq n} x_i.
    \end{equation*}
    %
    Taking the limit as $n \to \infty$, continuity of (binary) joins implies that $x_k \join s = s$, or $x_k \leq s$. Thus $s$ is an upper bound of the sequence $(x_n)$.

    On the other hand, if $t \in \altoverline{B}$ is an upper bound of $(x_n)$, then $x_k \leq t$. We have just seen that taking limits preserves inequalities, so this implies that $s \leq t$ as desired.

    The corresponding result for meets follows similarly, or from the fact that complementation is continuous.
\end{proof}


\section[Abstract sigma-algebras and continuity][Abstract $\sigma$-algebras and continuity]{Abstract $\sigma$-algebras and continuity}

\begin{definition}[Axiom of continuity]
    A generalised abstract measure space $(B, \mu)$ is said to satisfy the \emph{axiom of continuity} if it has the following property: If $(x_n)_{n\in\naturals}$ is a decreasing sequence of elements in $B$ such that $\bigmeet_{n\in\naturals} x_n$ exists and equals $0$, then $\lim_{n\to\infty} \mu(x_n) = 0$.
\end{definition}
%
This is an analogue of the continuity (from above) of ordinary countably additive measures on concrete $\sigma$-algebras.


\begin{lemma}
    \label{thm:positive-definite-implies-continuous}
    Let $(B, \mu)$ be a generalised abstract measure space with $\mu$ positive definite. Then $(B,\mu)$ satisfies the axiom of continuity.
\end{lemma}

\begin{proof}
    Let $(x_n)_{n\in\naturals}$ be a decreasing sequence in $B$ such that $\bigmeet_{n\in\naturals} x_n = 0$. Since it is decreasing, $x_n = \bigmeet_{i \leq n} x_i$, so because $\mu$ is positive definite, \cref{thm:existence-of-joins} implies that the sequence converges to its meet, i.e. it converges to $0$. By \cref{rem:convergence-of-measure}, $\mu(x_n)$ converges to $\mu(0) = 0$, proving the claim.
\end{proof}


\begin{proposition}[The generalised addition theorem]
    Let $(B, \mu)$ be a generalised abstract measure space satisfying the axiom of continuity. If $(x_n)_{n\in\naturals}$ is a sequence of pairwise disjoint elements in $B$ such that $x = \bigjoin_{n\in\naturals} x_n \in B$, then
    %
    \begin{equation*}
        \mu(x)
            = \sum_{n=1}^\infty \mu(x_n).
    \end{equation*}
\end{proposition}

\begin{proof}
    By \cref{thm:refine_join}, $r_n = \bigjoin_{i > n} x_i$ exists in $B$, and we claim that $\bigmeet_{n\in\naturals} r_n = 0$. Let $t \in \calF$ be a lower bound of $r_n$ for $n \in \naturals$. Then for $n \in \naturals$ we have
    %
    \begin{equation*}
        t \join \bigjoin_{i > n} x_i
            = \bigjoin_{i > n} x_i,
    \end{equation*}
    %
    and taking the meet of each side with $x_n'$ yields $t \meet x_n' = 0$. Hence $\bigjoin_{n\in\naturals} t \meet x_n = 0$. Now notice that $\bigmeet_{n\in\naturals} x_n = 0$, since $x_n \meet x_m = 0$ when $n \neq m$. It follows by taking complements that $\bigjoin_{n\in\naturals} x_n' = 1$, and so
    %
    \begin{equation*}
        t
            = t \meet \bigjoin_{n\in\naturals} x_n'
            = \bigjoin_{n\in\naturals} t \meet x_n'
            = 0.
    \end{equation*}
    %
    Thus $\bigmeet_{n\in\naturals} r_n = 0$ as claimed. It now follows from finite additivity of $\mu$ and the axiom of continuity that
    %
    \begin{equation*}
        \mu(x)
            = \sum_{i=1}^n \mu(x_i) + \mu(r_n)
            \to \sum_{i=1}^\infty \mu(x_i)
    \end{equation*}
    %
    as $n \to \infty$ as desired.
\end{proof}


Of course, the join of a sequence of elements may not exist. In the case where we are ensured the existence of countable joins we use the following terminology:

\begin{definition}[Abstract $\sigma$-algebra]
    An \emph{abstract $\sigma$-algebra} is a Boolean algebra $B$ with countable joins. That is, if $(x_n)_{n\in\naturals}$ is a sequence of elements in $B$, then their join $\bigjoin_{n\in\naturals} x_n$ exists.
\end{definition}
%
If the join $\bigjoin_{n\in\naturals} x_n$ exists, then it follows by taking complements that the meet $\bigmeet_{n\in\naturals} x_n'$ also exists. Hence an abstract $\sigma$-algebra also has countable meets. In the context of abstract measure spaces we obtain the following:

\begin{definition}[Abstract measure spaces]
    An \emph{abstract measure space} is a generalised abstract measure space $(B,\mu)$ that satisfies the axiom of continuity, and where $B$ is an abstract $\sigma$-algebra.
\end{definition}

\begin{lemma}
    If $(B,\mu)$ is a generalised abstract measure space with $\mu$ positive definite, then the completion $(\altoverline{B},\altoverline{\mu})$ is an abstract measure space.
\end{lemma}

\begin{proof}
    Since the completion of a metric space is a metric space, $\altoverline{\mu}$ is positive definite, so \cref{thm:positive-definite-implies-continuous} implies that $(\altoverline{B},\altoverline{\mu})$ satisfies the axiom of continuity.

    On the other hand, \cref{thm:existence-of-joins} implies that every sequence in $\altoverline{B}$ has a join, so $\altoverline{B}$ is an abstract $\sigma$-algebra.
\end{proof}



\chapter{The algebra of probability spaces}

\section{Motivation}

\newcommand{\compl}[1]{\altoverline{#1}}

If the probability of an event is supposed to be a measure of how often this event occurs on repetitions of the random experiment in question, then it seems reasonable to assume that we are, at least in principle, able to distinguish when the event does and does not obtain. For example, after rolling a six-sided die the state of affairs \textquote{the result of the die roll is three} is an event, since we can determine the outcome of the roll just by looking at the die. To take another example, after throwing a ball the state of affairs \textquote{the ball was thrown more than 50 metres} is also an event: That is, we can determine whether or not the length of the throw was strictly greater than 50 metres.

One might take a different view: Say that one grants that it is possible to \emph{affirm} that the length of the throw, measured in metres, lies in the interval $(50,\infty)$. If the length $L$ in metres does in fact lie in the above interval, we can simply use a ruler whose subdivisions are smaller than $L - 50$ in metres. Still one might disagree that it is possible to \emph{refute} that $L \in (50, \infty)$. For if $L$ is exactly $50$ metres, then since any measurement of $L$ carries some error, it is in practice impossible to determine whether $L$ is $50$ (or slightly smaller), or whether it is slightly larger than $50$. We will not pursue this line further but refer the reader to \textcite{vickers1989} for more on this \emph{logic of affirmative assertions}.

To be precise, after performing the relevant random experiment, we will assume that we are always able to decide whether or not the event has occurred or not. In particular, if $E$ is an event, then the state of affairs \textquote{$E$ does not obtain} is also an event, denoted $\compl{E}$: If $E$ obtains, then $\compl{E}$ does not. And conversely, if $E$ does not obtain, then $\compl{E}$ does obtain. We call $\compl{E}$ the \emph{complement of} or the \emph{complementary event to $E$}.\footnote{In contrast, in the logic of affirmative assertions we do not allow complementation (i.e. negation). Hence it may not be surprising that this logic ends up being closely tied to topology, the relevant analogy being that the complement of an open set need not be open.} Evidently, the complement of a complement is just the event we started with.

Next consider two events $E_1$ and $E_2$. Since we are able to decide whether each of them have obtained, the same is true for the event \textquote{both $E_1$ and $E_2$ have obtained} and the event \textquote{at least one of $E_1$ and $E_2$ has obtained}. The first is called the \emph{conjunction} of $E_1$ and $E_2$ and is denoted $E_1 \meet E_2$, and the second is the \emph{disjunction} $E_1 \join E_2$ of $E_1$ and $E_2$.

Finally it seems natural to allow an \textquote{impossible event} $0$ which never occurs, as well as a \textquote{sure event} $1$ that always occurs. Clearly $0$ and $1$ are each other's complements. If $E_1$ and $E_2$ are events with $E_1 \meet E_2 = 0$, then this manifestly means that $E_1$ and $E_2$ cannot obtain simultaneously: The two events are \emph{incompatible}.

We collect all the relevant events in a set $\calF$ and postulate that the structure $\langle \calF; \join, \meet, \compl{\,\cdot\,\vphantom{e}}, 0, 1 \rangle$ is a Boolean algebra. We leave it to the reader to reflect on the reasonability of this assumption. This leads naturally to the following definition:

\begin{definition}[Generalised abstract probability spaces]
    A \emph{generalised abstract probability space} is a generalised abstract measure space $(\calF, \P)$, where $\P$ is a positive definite probability measure.
\end{definition}
%
The partial order $\leq$ induced by the lattice structure then has the interpretation that if $E \leq F$, then $E$ implies $F$. For recall that this means that $E \join F = F$, i.e. if $F$ has already occurred then no information is gained by observing that $E$ has also occurred. Conversely, if $F$ has \emph{not} occurred, then $E$ is impossible.

We have justified every part of this definition except for the positive definiteness of $\P$. In the usual measure theoretical formulation of probability theory, there may (and often do) exist events that are not empty but still have probability zero. Kolmogorov had the following to say in critique of this approach:
%
\blockquote[\cite{kolmogorov1995}]{%
    \textins*{W}e are forced to give up the principle, formulated in numerous classical works in probability theory, according to which an event of probability zero is absolutely impossible. More precisely, one must allow that an event of positive probability can be decomposed into a (possible continuous) infinity of variants of which each has probability
    zero.%
}
%
Furthermore, this also has the technical benefit that the pseudometric $\rho_\P$ induced by $\P$ is in fact a metric. We will return to the consequences of this choice later.


\section[Abstract sigma-algebras and continuity][Abstract $\sigma$-algebras and continuity]{Abstract $\sigma$-algebras and continuity}

Of course, the map $\P$ above is supposed to be analogous to a probability measure on a (concrete) $\sigma$-algebra. But ordinary measures are \emph{countably} additive, not just finitely so. It is however difficult to justify extending the finite additivity to sequences of disjoint events purely on conceptual or operational groups. In fact, according to Kolmogorov himself:
%
\blockquote[\cite{kolmogorov1956}]{%
    Since the new axiom \textins{countable additivity} is essential for infinite fields of probability only, it is almost impossible to elucidate its empirical meaning. \textelp{} For, in describing any observable random process we can obtain only finite fields of probability. Infinite fields of probability occur only as idealized models of real random processes. \emph{We limit ourselves, arbitrarily, to only those models which satisfy Axiom VI} \textins{countable additivity}. This limitation has been found expedient in researches of the most diverse sort.%
}
%
And furthermore:
%
\blockquote[\cite{kolmogorov1995}]{%
    \textins*{S}omewhat more complicated problems require, if the theory is to be simple and tractable, that probability be subject to the \emph{axiom of denumerable additivity}. However, the justification of that axiom remains purely empirical, in that we have not yet encountered any interesting problem for which we have not been able to construct a probability field conforming to the axiom in question.%
}
%
\textcite{kolmogorov1956} introduces the axiom in the following form:


\nocite{*}

\printbibliography


\end{document}